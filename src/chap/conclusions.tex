%!TEX root = ../thesis.tex

\chapter{Conclusions and Outlook}
\section{Conclusions}\label{sec:concl:conclusions}
\textit{\footnotesize Parts of the text in this section have already been submitted for publication to Physical Review Letters in D. Freire-Fernandez et al., Measurement of the Isolated Nuclear Two-Photon Decay in $^{72}\mathrm{Ge}$ (2024) \cite{freirefernández2023measurement}}.

%Conclusions of the thesis 
\lettrine{S}{ummarizing}, I reported the results of the combined \textbf{\textsc{Schottky + Isochronous Mass Spectrometry (S+IMS)}} applied to the direct determination of the excitation energy and partial half-life of an isomer in the millisecond regime with high precision. This represents a \underline{dramatic extension} of the storage-ring based non-destructive lifetime spectroscopy to shorter-lived species as compared to previous experiments with electron-cooled stored beams\,\cite{Litvinov-2011}. A \textbf{mass resolving power} of $\bm{9.1\times10^5}$ (refer to \cref{sec:chap3:massresolvingpower}) has been achieved, which allows us to fully resolve $\bm{\sim100}$\,\textbf{keV} excited states, as was the case of \,\isomer{72}{\textbf{Br}} (showed in \cref{sec:chap3:massresolvingpower}). For the $\bm{2\gamma}$ \textbf{decay} of the $0^+$ isomer in $\bm{^{72}\mathrm{Ge}}$, a \textbf{partial half-life} of \,$\bm{\overline{T}_{1/2}^{\mathrm{rest}} = 23.9\left(6\right)}$\,\textbf{ms} (\cref{subsec:chap3:hl-twophoton}) and an excitation energy of $\bm{\omega=692.8\left(19\right)}$\,\textbf{keV} (\cref{subsec:chap3:72gemeasurementenergy}) have been determined.
The \underline{obtained partial half-life} is a \textbf{factor} $\bm{\sim10}$ \textbf{shorter than expected} from the extrapolation of previous results based on 
doubly-magic $^{16}\mathrm{O}$, $^{40}\mathrm{Ca}$ and semi-magic $^{90}\mathrm{Zr}$, indicating that the \underline{cancellation effect}, which arises from the non-positive products among nuclear matrix elements (in \cref{eq:chap1:alphaEL} and \cref{eq:chap1:chip}) observed in those nuclides, \underline{does not seem to occur}.

Additionally, we presented the (\textbf{preliminary}) \textbf{most precise} mass measurements obtained at storage rings (refer to \cref{sec:chap3:massmeasurements}). Highlights include \textbf{consistently} achieving an \textbf{uncertainty} of approximately $\bm{10}$\,\textbf{keV}, the potential \textbf{improvement} of the mass uncertainty of \,\isotope{\bm{69}}{\textbf{As}}\, \underline{by a third} of the \textsc{AME2020} value, and the observation of a \textbf{deviation} of about $\bm{30}$\,\textbf{keV} from the \textsc{AME2020} value for \,\isotope{\bm{72}}{\textbf{As}}\,. These \textbf{preliminary} results will be further discussed and published separately.

I also provided a \underline{mathematical} (refer to \cref{apdx:harmonicoverlap}) and \underline{illustrative} (refer to \cref{sec:chap3:identification}) example of \textbf{harmonic overlapping} within the recorded Schottky data. In order to address the potential \textbf{contamination} caused by overlapping signals, I have developed a \textbf{new} off-line analysis technique. This method is designed for analyzing \textbf{unresolved} ions, enabling their precise mass and half-life measurements. In unresolved isomers, additionally, we could determine the isomeric ratio.
Furthermore, the observation of \textbf{pure single-photon decay} in bare ions allows us to determine \textbf{electron conversion factors} experimentally. This would allow the comparison between \underline{experimental} and, the usually tabulated, \underline{theoretical} electron conversion factors. Hence, this method could act as a validation tool for theoretical calculations, where observed deviations might lead to \textbf{new assignments} of the specific transition \textbf{multipolarities}. These results are also still \textbf{preliminary} and under discussion.

Due to the excellent results given by our technique, \textsc{S+IMS} could be applied to a multitude of experiments at storage rings. Some have been outlined in the list provided in \cref{sec:concl:potential}.

\section{Potential experiments}\label{sec:concl:potential}
In this section, I present a list of nuclides that could be investigated using our combined \textsc{Schottky plus Isochronous Mass Spectrometry} (\cref{chap:chap2:methodology}). This section first examines potential candidates for nuclear two-photon decay (\cref{subsec:concl:candidates}), followed by exploring the potential of our technique for searching undiscovered low-lying $E0$ (and $M0$) transitions (\cref{subsec:concl:undiscovered}).

\subsection{Further nuclear two-photon decay candidates}\label{subsec:concl:candidates}
\begin{itemize}
    \item \,\isotope{98}{Mo} is a stable nuclide with a ground state of $J^{\pi}=0^{+}$ \cite{A98}. It is characterized by having a first excited state $0^{+}$ at $734.75(4)$\,keV and $T_{1/2}=21.8(9)$\,ns \cite{A98} making it an excellent candidate for measuring the (non-competitive\footnote{Unlike gamma spectroscopy, which measures these tiny decay branches in competition with other decay processes (competitive), our technique (refer to \cref{sec:chap1:twophotonstorage}) relies on prohibiting the other decay channels. In such cases, we exclusively observe the nuclear two-photon decay, making it non-competitive or isolated.}) nuclear two-photon decay. A measurement campaign in this nuclide is scheduled for May $2024$, within the experiment \textsc{E018}~\cite{E018proposal} at \textsc{GSI}.
    \item \,\isotope{98}{Zr}, with a ground state of $J^{\pi}=0^{+}$ and a half-life of $30.7(4)$\,s decaying primarily via $\beta^{-}$ \cite{A98}, contains a first excited state $0^{+}$ at $854.06(6)$\,keV and with $T_{1/2}=64(7)$\,ns \cite{A98}. These features qualify this nuclide as a promising candidate for measuring the (isolated) nuclear two-photon decay. In May $2024$, experiment \textsc{E018}~\cite{E018proposal} will be conducted, aiming to measure the nuclear two-photon decay in this nuclide. 
    \item Nowadays, with the aid of high-resolution gamma spectrometers like \textsc{AGATA} \cite{AGATA}, it may be possible to measure the (competitive) nuclear two-photon decay in $^{72}\mathrm{Ge}$, since its branching ratio is higher than expected: $T_{1/2}/T_{1/2}^{\gamma\gamma} \approx 2\times10^{-5}$ \cite{freirefernández2023measurement}. A successful measurement would offer additional insights into the magnitudes and signs of the nuclear matrix elements in \cref{eq:chap1:2gd}, allowing for a direct comparison between theoretical predictions and experimental data. 
    Initial efforts in this direction have recently taken place at \textsc{HI$\gamma$S} \cite{higs72geproposal} and the University of Cologne.
    \item \,\isotope{72}{Kr} features both ground and first excited states with a $J^\pi=0^{+}$ \cite{A72}. The excited state, situated at $671(1)$\,keV and having a half-life of $26.3(21)$\,ns \cite{A72}, makes it an ideal candidate for the measurement of its (non-competitive) nuclear two-photon decay and exploring its shape-coexistence \cite{PhysRevLett.90.082502}. 
    \item \,\isotope{182}{Hg} features a $0^+$ ground state and, potentially, a first excited state $0^+$ at $\sim328$\,keV \cite{A182}. From the relation contained in \cref{eq:chap1:2gd} and illustrated in \cref{fig:chap1:extrapolation}, we would expect an extremely hindered $2\gamma$ decay branch, making it unreachable by any other technique, but the one described in this thesis. This makes \,\isotope{182}{Hg} a perfect candidate for exploring the nuclear two-photon decay \cite{A182}.
    \item \,\isotope{186}{Pb} stands out as potentially the only nuclide with its two first excited states and the ground state all being $0^+$ \cite{A186}. Therefore, when fully stripped, our technique could uniquely enable the simultaneous measurement of two nuclear two-photon decays!
    \item \,\isotope{190}{Pb} presents a first excited state at $658(4)$\,keV and a ground state, both $0^+$ \cite{A190}. A $0^+\rightarrow0^+$ transition with a half-life of $\leq 0.22$\,ns has been observed in its neutral state \cite{A190}. Thus, in the absence of electrons, our approach could potentially enable observation of the nuclear two-photon decay.
    \item \,\isotope{192}{Pb}, possessing a first excited state at $768.84(23)$\,keV \cite{A192} with $J^\pi=0^+$, as its ground state, is another potential candidate for measuring the nuclear $2\gamma$ decay with bare ions stored in rings.
    \item \,\isotope{194}{Pb} is a special nucleus. It features a first excited $0^+$ state at $930.70(21)$\,keV, with the ground state also being a $0^+$ state. This makes it a promising candidate for observing the nuclear two-photon decay. What elevates even more its uniqueness, is the potential to observe the bound-state electron-positron pair creation \cite{boundelectronpositron}. The underlying concept is that, although the excited state is below the pair-creation threshold, capturing an electron into the vacant atomic K-shell introduces an additional (atomic binding) energy of $101.3$\,keV to the reaction. Therefore, if the electron from the created pair is captured (bound), it would effectively increase the total energy available to $\approx 930.7 + 101.3 > 1022$~keV, thus enabling bound-state electron-positron pair creation.
\end{itemize}

\subsection{Search for undiscovered low-lying $E0$ (and $M0$) transitions}\label{subsec:concl:undiscovered}
This technique presents potential for discovering $E0$ (and $M0$) transitions that have eluded detection through conventional electron spectroscopy. By utilizing bare ions, some states can be prevented from decaying via their typical pathways, as introduced in \cref{sec:chap1:twophotonstorage} and proved in \cref{subsec:chap3:hl-twophoton}. Under these circumstances, the excited state is ``trapped'' for an extended duration, thereby facilitating its measurement. This approach could be applied to nuclei theoretically or empirically predicted to possess a first excited low-lying $J=0$ state alongside a $J=0$ ground state. An example of such an application was the search of such a state in \,\isotope{70}{Se}, where despite our efforts, no indications of such a state were observed.

\section{Future facilities}
Many of the nuclides listed in \cref{sec:concl:potential} for possible experiments are challenging to produce in sufficient quantities for measurement at existing facilities (refer to \cref{subsubsec:intro:srings}). However, by the end of this decade, two main heavy-ion research facilities are expected to come online: the \textsc{Facility for Antiproton and Ion Research} (\textsc{FAIR}) (refer to \cite{SPILLER2006305}), and the \textsc{High Intensity heavy ion Accelerator Facility} (\textsc{HIAF}) \cite{YANG2013263}. These facilities are designed to achieve higher beam energies and intensities, among others, facilitating the production of (exotic) (heavy) neutron and proton-rich nuclides. Moreover, they are expected to be equipped with multiple storage rings specifically designed for operating the isochronous mode, unlike the present ones, and integrated with various Schottky detectors. These features make the novel methodology described in \cref{chap:chap2:methodology} particularly promising to be further developed and applied.

\begin{center}
    \vspace*{1cm}
    \pgfornament[width=0.8cm, color=darkred]{5}
    %\pgfornament[width=5cm, color=darkred]{71}
    %\pgfornament[width=5cm, color=darkred]{71}
    \vspace*{\fill}
\end{center}