%!TEX root = ../thesis.tex

\subsection*{\normalsize \underline{Erste Messungen des nuklearen Zwei-Photonen-Zerfalls an Speicherringen}}
\textbf{Zusammenfassung} - In dieser Arbeit wird ein neuartiger Ansatz vorgestellt, bei dem Schottky- und Isochron-Massenspektrometrie (S+IMS) kombiniert werden, um die nukleare Zwei-Photonen- oder Doppel-Gamma-Zerfallsrate ($2\gamma$) im Niederenergiebereich unterhalb der Elektron-Positron-Paarbildungsschwelle direkt zu messen. Diese Methode erweitert die zerstörungsfreie Lebensdauerspektroskopie zu kürzerlebigen Spezies ($\gtrsim 1$\,ms) und erzielt ein mit IMS bisher noch nie erreichtes Massenauflösungsvermögen von \,$9.1\times10^5$. Dank dieser Fähigkeit konnten wir erfolgreich angeregte Zustände mit einer Energie von bis zu $101$\,keV, wie der des Isomers in $^{72}\mathrm{Br}$ auflösen.
Wir haben die Halbwertszeit für den $2\gamma$-Zerfall des ersten angeregten $0^+$-Zustands in reinen $^{72}\mathrm{Ge}$-Ionen mit $23.9\left(6\right)$\,ms bestimmt, ein Ergebnis, das erheblich von den bisherigen Erwartungen abweicht. Diese Abweichung ergibt sich möglicherweise aus der Struktur dieses Kernes im Vergleich zu den Strukturen der bisher gemessenen magischen und doppeltmagischen Kerne. Weitere Untersuchungen sind erforderlich, um diese Diskrepanzen vollständig zu verstehen.
\newpar
Darüber hinaus präsentiert diese Arbeit einige der präzisesten Massenmessungen, die jemals an Speicherringen durchgeführt wurden. Wir haben insbesondere die Massenunsicherheit für \,\isotope{69}{As} verbessert und eine signifikante Abweichung ($>3\sigma$) für \,\isotope{72}{As} von den zuvor angegebenen Werten festgestellt. 
\newpar
Diese Ergebnisse zeigen das Potenzial unserer Technik für Massenmessungen und für die Erforschung bisher unzugänglicher nuklearer Zerfallsprozesse und eröffnen dadurch neue Möglichkeiten für künftige Forschungen an Speicherringen.
\subsection*{\normalsize \underline{First nuclear two-photon decay measurements at storage rings}}
\textbf{Summary} - This thesis introduces a pioneering approach by combining Schottky and Isochronous Mass Spectrometry (S+IMS) to directly measure the nuclear two-photon or double-gamma ($2\gamma$) decay rate in the low-energy regime, below the electron-positron pair creation threshold. This method extends non-destructive lifetime spectroscopy to include shorter-lived species ($\gtrsim 1$\,ms), and achieves an unprecedented mass resolving power of \,$9.1\times10^5$ for IMS. This capability allowed us to successfully resolve excited states, down to the $101$\,keV isomer in $^{72}\mathrm{Br}$.
We determined the half-life for the $2\gamma$ decay of the first-excited $0^+$ state in bare $^{72}\mathrm{Ge}$ ions to be $23.9\left(6\right)$\,ms, a finding that significantly diverges from prior expectations. This divergence potentially results from the structure of this mid-shell nucleus in comparison to the structures of the so far measured semi-magic and doubly-magic nuclei. Further investigations are required to fully understand these discrepancies.
\newpar
Furthermore, this work presents some of the most precise mass measurements ever achieved at storage rings. We notably improved the mass uncertainty for \,\isotope{69}{As} and identified a significant deviation ($>3\sigma$) for \,\isotope{72}{As} from previously tabulated values. 
\newpar
These results demonstrate the potential of our technique for mass measurements and for exploring nuclear decay pathways previously inaccessible, opening new venues for future research at storage rings.
