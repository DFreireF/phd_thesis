%! TEX root = ../thesis.tex

\documentclass[
  draft,%
  twoside,
  12pt,
  a4paper,
  openright,
  physicsastro,
  english,
  marginleft=1in,
  marginright=1in,
  margintop=1in,
  marginbottom=1in,
  bindingoffset=6mm
]{hdphdthesis}
\usepackage[english]{babel}
\usepackage{afterpage}
\usepackage{tikz} %For the boxes
\usepackage{varwidth}
\usepackage{amsmath}
\usepackage{amssymb}
\usepackage{mathtools}
\usepackage{lipsum}
\usepackage{lastpage}
\usepackage{epigraph}
\setlength\epigraphwidth{.6\textwidth}
\setlength\epigraphrule{0pt}
% graphics
\usepackage{GoudyIn}
\usepackage{lettrine}
\usepackage[compat=1.1.0]{tikz-feynman} % for feynman diagrams
\usepackage[object=vectorian]{pgfornament}%ornaments
\usepackage{subfig}
\usepackage{graphicx}
  \graphicspath{{img/}}
\usepackage{xcolor}
\usepackage{color}
\definecolor{darkred}{RGB}{139,0,0}
\definecolor{darkdarkred}{RGB}{102,0,0}

\usepackage{colortbl}
\usepackage{tikz}
\usepackage{cancel}
\usepackage{multicol}
\usepackage{titlesec}
\usepackage{tabularx}
\usepackage{bm}
%Verbatim
\usepackage{fancyvrb}
\usepackage{tcolorbox}
\usepackage{minted}
\tcbuselibrary{skins, breakable, minted, xparse}
\newtcolorbox{pythoncode}[1][]{
    enhanced,
    breakable,
    colback=blue!5!white,
    colframe=blue!75!black,
    boxrule=0.5mm,
    arc=2mm,
    top=3mm,
    bottom=3mm,
    left=5mm,
    right=5mm,
    title=#1,
    fonttitle=\bfseries
}

\definecolor{bg}{gray}{0.95} % Background color for the code area

\DeclareTCBListing{mintedbox}{O{}m!O{}}{
  breakable=false,
  listing engine=minted,
  listing only,
  minted language=#2,
  minted style=default,
  minted options={
    linenos,
    gobble=0,
    breaklines=true,
    breakafter=,,
    fontsize=\small,
    numbersep=8pt,
    #1},
  boxsep=0pt,
  left skip=0pt,
  right skip=0pt,
  left=25pt,
  right=0pt,
  top=3pt,
  bottom=3pt,
  arc=5pt,
  leftrule=1pt,
  rightrule=1pt,
  bottomrule=2pt,
  toprule=2pt,
  colback=bg,
  colframe=orange!70,
  enhanced,
  overlay={
    \begin{tcbclipinterior}
    \fill[orange!20!white] (frame.south west) rectangle ([xshift=20pt]frame.north west);
    \end{tcbclipinterior}
  },
label=#3}
% tables
%--------------- Table and figure captions ----------------
%\usepackage[hyphens]{url}
\captionsetup[figure]{labelfont={bf, it},name={Figure},labelsep=period,singlelinecheck=true}
\captionsetup[table]{labelfont={bf, it},name={Table},labelsep=period,singlelinecheck=true}
\usepackage[%
hypertexnames=false,
pdfauthor={David Freire Fernández},
pdftitle={First nuclear two-photon decay measurements at storage rings},
pdfkeywords={physics, two-photon, storage rings},
pdfproducer={XeLaTeX with hyperref},
pdfcreator={XeLaTeX}
]{hyperref}
%\hypersetup{
%    pdfpagelayout={TwoPageRight},
%}
\usepackage{pdfpages}
\usepackage{xurl} 
\usepackage{booktabs}
\usepackage{makecell}
\usepackage{siunitx}
\usepackage{array}
\usepackage{threeparttable}
% bibliography, references
\usepackage{csquotes}
\usepackage[
  sorting=none,
  %maxnames=1,
  date=year,
  doi=true,
  url=true,
  eprint=true,
  isbn=false,
  style=numeric-comp
]{biblatex}
\addbibresource{bib/bibliography.bib}
\addbibresource{bib/2p_exp.bib}
\addbibresource{bib/2p_paper.bib}
\addbibresource{bib/2p_th.bib}
\addbibresource{bib/datasets.bib}
\addbibresource{bib/intro.bib}
\addbibresource{bib/chap2.bib}
\addbibresource{bib/chap1.bib}
\addbibresource{bib/cmm.bib}
\addbibresource{bib/conclusions.bib}
% remove In: and full stop at the end
\renewbibmacro{in:}{}
\renewcommand{\finentrypunct}{}

% remove title and issue number
\AtEveryBibitem{%
  \clearfield{title}
  \clearfield{issue}
}

% add link to doi.org
\DeclareFieldFormat{doi}{%
   \mkbibacro{DOI}\addcolon\space%
   \ifhyperref{\href{https://doi.org/#1}{\nolinkurl{#1}}}{\nolinkurl{#1}}
 }

% add collaboration, if there
\DeclareSourcemap{%
 \maps[datatype=bibtex, overwrite=true]{%
  \map{%
    \step[fieldsource=Collaboration, final=true]
    \step[fieldset=usera, origfieldval, final=true]
  }
 }
}
\renewbibmacro*{author}{%
  \iffieldundef{usera}{%
    \printnames{author}%
  }{%
    \printnames{author} (\textsc{\printfield{usera}} Collaboration)%
  }%
}%

% the following fix makes biblatex use UrlFont also for primaryClass field
\makeatletter
\DeclareFieldFormat{eprint:arxiv}{%
  \textsc{arXiv}\addcolon\space%
  \ifhyperref{%
    \href{https://arxiv.org/\abx@arxivpath/#1}{%
      \nolinkurl{#1}%
      \iffieldundef{eprintclass}
        {}
        {\addspace\UrlFont\mkbibbrackets{\thefield{eprintclass}}}
    }
  }{%
    \nolinkurl{#1}%
    \iffieldundef{eprintclass}
      {}
      {\addspace\mkbibbrackets{\thefield{eprintclass}}}
  }
}
\makeatother
\setcounter{biburllcpenalty}{9000}
\setcounter{biburlucpenalty}{9000}
\setcounter{biburlnumpenalty}{9000}
%\def\UrlBreaks{\do\/\do-}
\usetikzlibrary{calc}
\newcommand{\myendchapter}{%
    \begin{tikzpicture}[baseline=-0.5ex]
        % Draw the left part of the shape
        \filldraw[fill=black!50!red] (0,0) -- (0.5,1.5) -- (1,1.5) -- (1.5,0) -- cycle;
        % Draw the right part of the shape
        \filldraw[fill=black!50!red] (1.5,0) -- (2,1.5) -- (2.5,1.5) -- (3,0) -- cycle;
        % Draw the middle part of the shape
        \filldraw[fill=white] (1,1.5) -- (1.5,0) -- (2,1.5) -- cycle;
        % Draw the center circle
        \draw[fill=black!50!red] (1.5,0.75) circle (0.3cm);
    \end{tikzpicture}%
}
\newcommand{\mybox}[4][\textwidth-\pgfkeysvalueof{/pgf/inner xsep}-2mm]{
\begin{figure}[h]
\centering
\begin{tikzpicture}
\node[line width=.5mm, rounded corners, draw=#2, inner ysep=10pt, text width=#1, outer sep=0] (one) {\vspace*{15pt}\\\begin{varwidth}{\textwidth}#4\end{varwidth}};
\node[text=white,anchor=north east,align=center, minimum height=20pt] (two) at (one.north east) {#3 \hspace*{.5mm}};
\path[fill=#2]
    (one.north west|-two.west) --
    ($(two.west)+(-1.5cm,0)$)
    to[out=0,in=180] (two.south west) --
    (two.south east) [rounded corners] --
    (one.north east) -- 
    (one.north west) [sharp corners] -- cycle;
\node[text=white,anchor=north east,align=center, minimum height=20pt, text height=2ex] (three) at (one.north east) {#3 \hspace*{.5mm}};
\end{tikzpicture}
\end{figure}
}
\newcommand{\chapterendfigure}[1]{%
    %\clearpage
    \begin{figure}[htbp]
        \centering
        \includegraphics[width=0.1\textwidth]{#1} % Change width if needed
    \end{figure}
}

\newcommand{\floralheart}{%
    \par\nobreak
    \begin{center}
    \begin{tikzpicture}[scale=0.5]
    \draw[fill=red] (0,0) .. controls +(0,2) and +(2,0) .. (2,2) .. controls +(2,0) and +(0,2) .. (4,0) .. controls +(0,-2) and +(-2,0) .. (2,-2) .. controls +(-2,0) and +(0,-2) .. (0,0);
    \draw[fill=green!60!black] (2,0) circle (0.75);
    \foreach \i in {0,60,...,300} {
        \draw[fill=green!60!black] (2,0) -- ++(\i:0.75) circle (0.25);
    }
    \draw[fill=green!60!black] (2,0) circle (0.1);
    \end{tikzpicture}
    \end{center}
    \vskip 20pt
}
\usepackage{marginnote}\renewcommand*{\marginfont}{\scshape}
\newcommand{\blocktitle}[1]{\leavevmode\marginnote{#1}\ignorespaces}

\newcommand{\myboxfig}[4][\textwidth-\pgfkeysvalueof{/pgf/inner xsep}-2mm]{
    \begin{figure}[hbt]
        \centering
        \begin{tikzpicture}
            \node[line width=.5mm, rounded corners, draw=#2, inner ysep=10pt, text width=#1, outer sep=0] (one) {\vspace*{15pt}\\\begin{varwidth}{\textwidth}#4\end{varwidth}};
            \node[text=white,anchor=north east,align=center, minimum height=20pt] (two) at (one.north east) {#3 \hspace*{.5mm}};
            \path[fill=#2]
                (one.north west|-two.west) --
                ($(two.west)+(-1.5cm,0)$)
                to[out=0,in=180] (two.south west) --
                (two.south east) [rounded corners] --
                (one.north east) -- 
                (one.north west) [sharp corners] -- cycle;
            \node[text=white,anchor=north east,align=center, minimum height=20pt, text height=2ex] (three) at (one.north east) {#3 \hspace*{.5mm}};
        \end{tikzpicture}
    \end{figure}
}

\newcommand{\myboxfigdoublecolumn}[4][\textwidth-\pgfkeysvalueof{/pgf/inner xsep}-2mm]{
    \centering
    \begin{tikzpicture}
        \node[line width=.5mm, rounded corners, draw=#2, inner ysep=10pt, text width=#1, outer sep=0] (one) {\vspace*{15pt}\\\begin{varwidth}{\textwidth}#4\end{varwidth}};
        \node[text=white,anchor=north east,align=center, minimum height=20pt] (two) at (one.north east) {#3 \hspace*{.5mm}};
        \path[fill=#2]
            (one.north west|-two.west) --
            ($(two.west)+(-1.5cm,0)$)
            to[out=0,in=180] (two.south west) --
            (two.south east) [rounded corners] --
            (one.north east) -- 
            (one.north west) [sharp corners] -- cycle;
        \node[text=white,anchor=north east,align=center, minimum height=20pt, text height=2ex] (three) at (one.north east) {#3 \hspace*{.5mm}};
    \end{tikzpicture}
}

%\usepackage{fontspec} % This requires XeLaTeX
%\setmainfont{Times New Roman}
%fonts
\usepackage[lining]{ebgaramond}
%------------- Setting the license of thesis --------------
\usepackage[
    type={CC},
    modifier={by-nc-nd},
    version={3.0},
]{doclicense}

%cleverref
\usepackage[capitalise]{cleveref}
  \crefformat{chapter}{#2Chap.~#1#3}
  \crefformat{appendix}{#2App.~#1#3}
  \crefformat{section}{#2§#1#3}
  \crefformat{subsection}{#2§#1#3}
  \crefformat{subsubsubsection}{#2subsubsection~#1#3}  
  \crefformat{equation}{#2Eq.~(#1)#3}
  \crefformat{figure}{#2Fig.~#1#3}
  \crefformat{table}{#2Tab.~#1#3}
  \crefformat{minted}{#2Minted.~#1#3}

  \Crefformat{table}{#2Table~#1#3}
  \Crefformat{figure}{#2Figure~#1#3}
  \Crefformat{equation}{#2Equation~(#1)#3}
  \Crefformat{chapter}{#2Chapter~#1#3}
  \Crefformat{appendix}{#2Appendix~#1#3}
  \Crefformat{section}{#2Section~§#1#3}


%macros
%!TEX root = ../thesis.tex

\newcommand{\GSI}        {{\mbox{\textsc{GSI }}}}
\newcommand{\etal}       {{\it et al.}}
\newcommand{\gt}         {$\gamma_t$ }
\newcommand{\germanium}  {$^{72}\mathrm{Ge}$ }
\newcommand{\ezerotrans}    {$0^{+}\rightarrow0^{+}$ }
\newcommand{\masstocharge}    {$m/q$ }
\newcommand{\selenium}  {$^{70}\mathrm{Se}$ }
\newcommand{\Eonefourthree}       {\textsc{E143 }}
\newcommand{\gsi}       {\textsc{GSI }}
\newcommand{\esr}       {\textsc{ESR }}
\newcommand{\SMS}       {\textsc{SMS }}
\newcommand{\IMS}       {{\mbox{\textsc{IMS }}}}

\newcommand{\newpar}{\newline\indent}

\newcommand{\ion}[3]{$^{#1}\mathrm{#2}^{#3+}$}
\newcommand{\isotope}[2]{$^{#1}\mathrm{#2}$}
\newcommand{\isomer}[2]{$^{#1\mathrm{m}}\mathrm{#2}$}

\renewcommand{\LettrineFontHook}{\color{darkred}\GoudyInfamily{}}
\LettrineTextFont{\itshape}
\setcounter{DefaultLines}{3}%
%This is for the double lines in the chapter title
\usepackage{xpatch}
\xpretocmd\headrule{\color{darkdarkred}}{}{\PatchFailed}

\usepackage{microtype}

% diff:
\newcommand{\diff}[2]{\frac{d #1}{d #2}}
%\renewcommand{\thechapter}{\Roman{chapter}}
% Fancy chapter title format
\titleformat{\chapter}[display]
  {\bfseries\huge\color{black}} % Chapter title font settings
  {\centering\Large\color{darkred}\pgfornament[width=1.6cm]{72}\color{black}\chaptertitlename~\thechapter~\color{darkred}\pgfornament[width=1.6cm]{73}} % Chapter prefix (e.g., "Chapter 1") settings
  {3ex} % Space between chapter prefix and title
  {\color{darkred}\titlerule\titlerule\titlerule\color{black}\vspace{1ex}\centering\itshape} % Title formatting and rule
  [\vspace{1ex}\color{darkred}\titlerule\titlerule\titlerule] % Rule after the title

% Section for the CV
\usepackage{orcidlink}
%\usepackage{fontawesome}
% Environment setup for rSection
\newsavebox{\titlebox}
\newlength{\sectionlineskip}
\newlength{\sectionskip}
\setlength{\sectionlineskip}{2pt} % Adjust the vertical space after the line
\setlength{\sectionskip}{10pt} % Adjust the vertical space before the section title
\usetikzlibrary{fadings}
\tikzfading[name=fade out in middle,
  left color=transparent!0, right color=transparent!0,
  middle color=transparent!100]
\newenvironment{rSection}[1]{ % 1 input argument - section name
  \vspace{\sectionskip}
  \savebox{\titlebox}{\bfseries\MakeUppercase{#1}} % Save the title in the box
  \settowidth{\titlewidth}{\usebox{\titlebox}} % Measure the width of the title
  \pgfmathsetlengthmacro{\lineend}{\linewidth-\titlewidth-0.em} % Calculate the remaining width
  
  \usebox{\titlebox} % Print the section title
  \vspace{\sectionlineskip}
  
  % Fading horizontal line under the section title
  \begin{tikzpicture}
    \begin{scope}
      \clip (0,-0.1) rectangle (\lineend,0.1);
      \shade[left color=black, right color=white] (0,0) rectangle (\lineend,0.1);
    \end{scope}
  \end{tikzpicture}
  
  \begin{list}{}{ % List for each individual item in the section
    \setlength{\leftmargin}{1.5em} % Margin within the section
  }
  \item[]
}{
  \end{list}
}
\usepackage{enumitem}
\setlist[itemize,1]{label={\bf \faAngleRight}, leftmargin=3em}
\setlist[itemize,2]{label={\bf \faAngleRight}, nosep, leftmargin=1.5em}
\usepackage{fontawesome5}